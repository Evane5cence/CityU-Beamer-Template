\section{Methodology}

\subsection{Subsection 1}

\begin{frame}{Frame Title}
    \begin{itemize}
        \item Modify aspect ratio at slide.tex line 1
        \item Removed Chinese (ctex) support
    \end{itemize}
\end{frame}

\subsection{Subsection 2}

\begin{frame}{Why Beamer}
    \begin{itemize}
        \item \LaTeX is widely used in academia
    \end{itemize}
    \begin{table}[h]
        \centering
        \begin{tabular}{c|c}
            Microsoft\textsuperscript{\textregistered}  Word & \LaTeX \\
            \hline
            Easy to use & Easy to use \\
            Paid & Free \\
        \end{tabular}
    \end{table}
\end{frame}

\begin{frame}{Formatting Examples}
    \begin{exampleblock}{Unnumbered Equation} % add * 
        \begin{equation*}
            J(\theta) = \mathbb{E}_{\pi_\theta}[G_t] = \sum_{s\in\mathcal{S}} d^\pi (s)V^\pi(s)=\sum_{s\in\mathcal{S}} d^\pi(s)\sum_{a\in\mathcal{A}}\pi_\theta(a|s)Q^\pi(s,a)
        \end{equation*}
    \end{exampleblock}
    \begin{exampleblock}{Numbered Equation \footnote{If there's word in the euqation, please use $\backslash$mathrm\{\} or $\backslash$text\{\}.}}
        % use & to separate
        \begin{align}
            Q_\mathrm{target}&=r+\gamma Q^\pi(s^\prime, \pi_\theta(s^\prime)+\epsilon)\\
            \epsilon&\sim\mathrm{clip}(\mathcal{N}(0, \sigma), -c, c)\nonumber
        \end{align}
    \end{exampleblock}
\end{frame}

\begin{frame}{Formatting Examples}
    \begin{exampleblock}{Numbered Multi-line Equation}
        % Taken from Mathmode.tex
        \begin{multline}
            A=\lim_{n\rightarrow\infty}\Delta x\left(a^{2}+\left(a^{2}+2a\Delta x+\left(\Delta x\right)^{2}\right)\right.\label{eq:reset}\\
            +\left(a^{2}+2\cdot2a\Delta x+2^{2}\left(\Delta x\right)^{2}\right)\\
            +\left(a^{2}+2\cdot3a\Delta x+3^{2}\left(\Delta x\right)^{2}\right)\\
            +\ldots\\
            \left.+\left(a^{2}+2\cdot(n-1)a\Delta x+(n-1)^{2}\left(\Delta x\right)^{2}\right)\right)\\
            =\frac{1}{3}\left(b^{3}-a^{3}\right)
        \end{multline}
    \end{exampleblock}
\end{frame}

\begin{frame}{Figures and Columns}
    % From thuthesis user guide.
    \begin{minipage}[c]{0.3\linewidth}
        \psset{unit=0.8cm}
        \begin{pspicture}(-1.75,-3)(3.25,4)
            \psline[linewidth=0.25pt](0,0)(0,4)
            \rput[tl]{0}(0.2,2){$\vec e_z$}
            \rput[tr]{0}(-0.9,1.4){$\vec e$}
            \rput[tl]{0}(2.8,-1.1){$\vec C_{ptm{ext}}$}
            \rput[br]{0}(-0.3,2.1){$\theta$}
            \rput{25}(0,0){%
            \psframe[fillstyle=solid,fillcolor=lightgray,linewidth=.8pt](-0.1,-3.2)(0.1,0)}
            \rput{25}(0,0){%
            \psellipse[fillstyle=solid,fillcolor=yellow,linewidth=3pt](0,0)(1.5,0.5)}
            \rput{25}(0,0){%
            \psframe[fillstyle=solid,fillcolor=lightgray,linewidth=.8pt](-0.1,0)(0.1,3.2)}
            \rput{25}(0,0){\psline[linecolor=red,linewidth=1.5pt]{->}(0,0)(0.,2)}
%           \psRotation{0}(0,3.5){$\dot\phi$}
%           \psRotation{25}(-1.2,2.6){$\dot\psi$}
            \psline[linecolor=red,linewidth=1.25pt]{->}(0,0)(0,2)
            \psline[linecolor=red,linewidth=1.25pt]{->}(0,0)(3,-1)
            \psline[linecolor=red,linewidth=1.25pt]{->}(0,0)(2.85,-0.95)
            \psarc{->}{2.1}{90}{112.5}
            \rput[bl](.1,.01){C}
        \end{pspicture}
    \end{minipage}\hspace{1cm}
    \begin{minipage}{0.5\linewidth}
        \medskip
        %\hspace{2cm}
        \begin{figure}[h]
            \centering
            \includegraphics[height=.4\textheight]{pic/dtmf.pdf}
        \end{figure}
    \end{minipage}
\end{frame}

\begin{frame}[fragile]{\LaTeX{} Frequently Used Commands}
    \begin{exampleblock}{Commands}
        \centering
        \footnotesize
        \begin{tabular}{llll}
            \cmd{chapter} & \cmd{section} & \cmd{subsection} & \cmd{paragraph} \\
            Chapter & Section & Subsection & Paragraph \\\hline
            \cmd{centering} & \cmd{emph} & \cmd{verb} & \cmd{url} \\
            Centering & Emphasis & Verbatim & URL \\\hline
            \cmd{footnote} & \cmd{item} & \cmd{caption} & \cmd{includegraphics} \\
            Footnote & Item & Caption & Graphics \\\hline
            \cmd{label} & \cmd{cite} & \cmd{ref} \\
            Label & Cite & Reference\\\hline
        \end{tabular}
    \end{exampleblock}
    \begin{exampleblock}{Environments}
        \centering
        \footnotesize
        \begin{tabular}{lll}
            \env{table} & \env{figure} & \env{equation}\\
            Table & Figure & Equation \\\hline
            \env{itemize} & \env{enumerate} & \env{description}\\
            Itemize & Enumerate & Description \\\hline
        \end{tabular}
    \end{exampleblock}
\end{frame}

\begin{frame}[fragile]{\LaTeX{} Env Examples}
    \begin{minipage}{0.5\linewidth}
\begin{lstlisting}[language=TeX]
\begin{itemize}
  \item A \item B
  \item C
  \begin{itemize}
    \item C-1
  \end{itemize}
\end{itemize}
\end{lstlisting}
    \end{minipage}\hspace{1cm}
    \begin{minipage}{0.3\linewidth}
        \begin{itemize}
            \item A
            \item B
            \item C
            \begin{itemize}
                \item C-1
            \end{itemize}
        \end{itemize}
    \end{minipage}
    \medskip
    \pause
    \begin{minipage}{0.5\linewidth}
\begin{lstlisting}[language=TeX]
\begin{enumerate}
  \item A \item B
  \item C
  \begin{itemize}
    \item[n+e] D
  \end{itemize}
\end{enumerate}
\end{lstlisting}
    \end{minipage}\hspace{1cm}
    \begin{minipage}{0.3\linewidth}
        \begin{enumerate}
            \item A
            \item B
            \item C
            \begin{itemize}
                \item[n+e] D
            \end{itemize}
        \end{enumerate}
    \end{minipage}
\end{frame}

\begin{frame}[fragile]{\LaTeX{} Math Equations}
    \begin{columns}
        \begin{column}{.55\textwidth}
\begin{lstlisting}[language=TeX]
$V = \frac{4}{3}\pi r^3$

\[
  V = \frac{4}{3}\pi r^3
\]

\begin{equation}
  \label{eq:vsphere}
  V = \frac{4}{3}\pi r^3
\end{equation}
\end{lstlisting}
        \end{column}
        \begin{column}{.4\textwidth}
            $V = \frac{4}{3}\pi r^3$
            \[
                V = \frac{4}{3}\pi r^3
            \]
            \begin{equation}
                \label{eq:vsphere}
                V = \frac{4}{3}\pi r^3
            \end{equation}
        \end{column}
    \end{columns}
    \begin{itemize}
        \item More examples, please visit \href{https://en.wikipedia.org/wiki/Help:Displaying_a_formula}{\color{purple}{here}}
    \end{itemize}
\end{frame}

% \begin{frame}[fragile]
%     \begin{columns}
%         \column{.6\textwidth}
% \begin{lstlisting}[language=TeX]
%     \begin{table}[htbp]
%       \caption{编号与含义}
%       \label{tab:number}
%       \centering
%       \begin{tabular}{cl}
%         \toprule
%         编号 & 含义 \\
%         \midrule
%         1 & 4.0 \\
%         2 & 3.7 \\
%         \bottomrule
%       \end{tabular}
%     \end{table}
%     公式~(\ref{eq:vsphere}) 的
%     编号与含义请参见
%     表~\ref{tab:number}。
% \end{lstlisting}
%         \column{.4\textwidth}
%         \begin{table}[htpb]
%             \centering
%             \caption{编号与含义}
%             \label{tab:number}
%             \begin{tabular}{cl}\toprule
%                 编号 & 含义 \\\midrule
%                 1 & 4.0\\
%                 2 & 3.7\\\bottomrule
%             \end{tabular}
%         \end{table}
%         \normalsize 公式~(\ref{eq:vsphere})的编号与含义请参见表~\ref{tab:number}。
%     \end{columns}
% \end{frame}

% \begin{frame}{作图}
%     \begin{itemize}
%         \item 矢量图 eps, ps, pdf
%         \begin{itemize}
%             \item METAPOST, pstricks, pgf $\ldots$
%             \item Xfig, Dia, Visio, Inkscape $\ldots$
%             \item Matlab / Excel 等保存为 pdf
%         \end{itemize}
%         \item 标量图 png, jpg, tiff $\ldots$
%         \begin{itemize}
%             \item 提高清晰度,避免发虚
%             \item 应尽量避免使用
%         \end{itemize}
%     \end{itemize}
%     \begin{figure}[htpb]
%         \centering
%         \includegraphics[width=0.2\linewidth]{pic/cityu_vertical_logo_cmyk.eps}
%         \caption{这个校徽就是矢量图}
%     \end{figure}
% \end{frame}
